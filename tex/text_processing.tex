\documentclass[a4j]{jarticle}
\usepackage{listings,jlisting}
\usepackage{color}
\usepackage{moreverb}
\usepackage{eclbkbox}
\usepackage{framed}
\usepackage[top=30truemm,bottom=30truemm,left=15truemm,right=15truemm]{geometry}
\definecolor{OliveGreen}{rgb}{0.0,0.6,0.0}
\lstset{
  basicstyle={\ttfamily},
  identifierstyle={\small},
  commentstyle={\smallitshape \color[rgb]{0,0.5,0}},
  keywordstyle={\small\bfseries},
  ndkeywordstyle={\small},
  stringstyle={\small\ttfamily},
  frame={tb},
  breaklines=true,
  columns=[l]{fullflexible},
  numbers=left,
  xrightmargin=0zw,
  xleftmargin=0zw,
  numberstyle={\scriptsize},
  stepnumber=1,
  numbersep=1zw,
  lineskip=-0.5ex,
  linewidth=15cm
}
%ここまでソースコードの表示に関する設定
\begin{document}

\subsection*{5.1}
  課題5\_1text1からtext2までの語彙数\\
  text1: 17231\\
  text2: 6403\\
  text3: 2628\\
  text4: 9201\\
  text5: 5441\\
  text6: 1855\\
  text7: 11387\\
  text8: 882\\
  text9: 6349
\subsection*{5.2}
  課題5\_2 アルファベットの出現率\\
  \fontsize{5.5pt}{0pt}\verbatimtabinput{./text/kadai5_2.txt}
\subsection*{5.3}
  課題5\_3 text1からtext9までの出現率上位5番目までの4文字以上の単語\\
  \fontsize{10pt}{5pt}\verbatimtabinput{./text/kadai5_3.txt}
  出現回数1位はthatであった。
\subsection*{5.4}
  課題5\_4 text1からtext9までのストップワードの比率\\
  \fontsize{10pt}{10pt}\verbatimtabinput{./text/kadai5_4.txt}
\subsection*{5.5}
  課題5\_5 wagahaiha\_nekodearu.txtに含まれる名詞の比率\\
  形態素解析を行いすべての品詞をすべて足したものを分母とし、分子は名詞の数とした。\\
  \fontsize{10pt}{10pt}\verbatimtabinput{./text/kadai5_5.txt}
\subsection*{5.6}
  課題5\_6 wagahaiha\_nekodearu.txtに含まれる出現頻度上位30位までの名詞\\
  \fontsize{10pt}{10pt}\verbatimtabinput{./text/kadai5_6.txt}
\subsection*{5.7}
  課題5\_7 wagahaiha\_nekodearu.txtにtop30\_jnounが含まれる頻度\\
  \fontsize{10pt}{10pt}\verbatimtabinput{./text/kadai5_7.txt}



\end{document}
